\documentclass{article}

\usepackage{amsthm}
\usepackage{amsfonts}
\usepackage{amsmath}
\usepackage{amssymb}
\usepackage{fullpage}
\usepackage[usenames]{color}
\usepackage{hyperref}
\hypersetup{
colorlinks = true,
urlcolor = blue, % color of external links using \href
linkcolor= blue, % color of internal links
citecolor= blue, % color of links to bibliography
filecolor= blue, % color of file links
}
\usepackage{listings}

\definecolor{dkgreen}{rgb}{0,0.6,0}
\definecolor{gray}{rgb}{0.5,0.5,0.5}
\definecolor{mauve}{rgb}{0.58,0,0.82}

\lstset{frame=tb,
language=haskell,
aboveskip=3mm,
belowskip=3mm,
showstringspaces=false,
columns=flexible,
basicstyle={\small\ttfamily},
numbers=none,
numberstyle=\tiny\color{gray},
keywordstyle=\color{blue},
commentstyle=\color{dkgreen},
stringstyle=\color{mauve},
breaklines=true,
breakatwhitespace=true,
tabsize=3
}


\title{CPSC-354 Report}
\author{Connor Cowher \\ Chapman University}

\date{10/17/2021}

\begin{document}

\maketitle

\begin{abstract}
This report will dive into the aspects of learning Haskell and the functions to know when creating a project. Haskell is much different from other programming languages like Java and C/C++ and requires a more extensive knowledge to perfect. This software has an extensive Prelude, similar to a library, that should be looked into in order to better utilize the functions for your program. Haskell is a great introductory language for new programmers since it is its own unique assembly type language. Although you do not need a masters degree in mathematics, it would be beneficial to have some exposure in advanced arithmetic. Haskell is both as difficult and simple as you make it. As long as you study, understand, and make good decisions while programming, Haskell will be a great platform for you to utilize. I will also be explaining how to make a project in this language, while showing some example code in Haskell. Coding in this language seems very basic and very few characters are required per line, for the most part. A very useful topic to know and have in your back pocket would be lambda calculus. This version of mathematics is very helpful in computer calculation logic, a major aspect of Haskell. While helping out in logic, another helpful aspect of lambda calculus is its ability to simulate on a Turing machine. Though limited to linux systems, a virtual machine, or virtual box application, can simulate an apple OS while on a windows machine...

\end{abstract}

\tableofcontents

\section{Introduction}\label{intro} 

Haskell is a programming language where it may take some time to understand all the features and nuances. One aspect where Haskell differs from the common universal languages is being "lazy". This means Haskell will not do any work unless it needs to be completed. It is as close to assembly language we have had in all of the courses Chapman offers. With learning Haskell, you do not necessarily have to have the most extensive programming experience or knowledge. Haskell is so vastly different from all other languages that it makes it the perfect starting point. Although learning something so different and unique has its advantages, being skilled in mathematics has its benefits to this language. Much of the beginning is understanding and interpreting basic mathematics in different formats. Something as simple as a basic calculator proves its challenges and will make you have an appreciation for everyday programs we use without an extra thought. Since most students taking this course are upperclassmen, that does imply a more extensive understanding than most individuals so we can assume this isn’t our first time learning a new language. Assembly language is vastly different from interpreter languages as one knows, but understanding how the integers and and negative numbers work with different operations will take some time. Firstly, functions are unique in the way they are created. There are multiple classes that must work hand in hand to make a program function correctly. The first assignment for Programming Languages was to create, from a template, a calculator and make our own operations. With a partner we concluded to do our basic 4 mathematical operations and added a couple more, commonly found on a TI-84 calculator. While these other operations, square root, squared, etc. are simple in theory, creating a function that can handle any type of input proves much harder to accomplish. With that out of the way we can dive into the beginning of starting this language. To begin, you should start downloading Haskell by visiting … (insert download link). 
After familiarizing yourself with the UI and paths needed to have interacting classes, a beneficial task to complete would be to go back and understand lambda calculus. A majority of functions will require some knowledge of lambda calculus and understanding it would fast track any issues you run into while coding any mathematical functions in Haskell. 


\subsection{General Remarks}
Everything below here is used for template purposes and will not be included in my report in its current form. ~~Work in progress below~~


First you need to \href{https://www.latex-project.org/get/}{download and install} LaTeX.\footnote{Links are typeset in blue, but you can change the layout and color of the links if you locate the \texttt{hypersetup} command.}
%
Alternatively, you can use an online editor such as \href{https://www.overleaf.com/learn}{Overleaf}. I prefer to have my own installation, but to get started Overleaf may be easier.

\medskip\noindent
LaTeX is a markup language (as is, for example, HTML). The source code is in a \verb+.tex+ file and needs to be compiled for viewing, usually to \verb+.pdf+.


\medskip\noindent
If you want to change the default layout, you need to type commands. For example, \verb+\medskip+ inserts a medium vertical space and \verb+\noindent+ starts a paragraph without indentation.
\medskip\noindent
Mathematics is typeset double dollars, for example $$x+y=y+x.$$


\subsection{LaTeX Resources}

I start a new subsection, so that you can see how it appears in the table of contents.

\begin{itemize}
\item This is how you itemize in LaTeX.
\item I think a good way to learn LaTeX is by starting from this template file and build it up step by step. Often stackoverflow will answer your questions. But here are a few resources:
\begin{enumerate}
\item \href{https://www.overleaf.com/learn/latex/Learn_LaTeX_in_30_minutes}{Learn LaTeX in 30 minutes}
\item \href{https://www.latex-project.org/}{LaTeX – A document preparation system}\end{enumerate}
\end{itemize}

\subsection{Plagiarism}

To avoid plagiarism, make sure that in addition to \cite{PL} you also cite all the external sources you use.

\section{Haskell}\label{haskell}

Learning a new programming language is almost always a steep learning curve. At first it is very difficult to understand and utilize all the nuances of the language. However, this is not the case with Haskell. Unlike every other interpreter language professors use here at Chapman, Haskell is vastly unique and takes a lot more time to fully understand how it works. For instance, Haskell is not an interpreter language, but actually a functional programming language. A bonus to being a functional programming language is the ability to change state whenever you desire. When you run a program in Java or C/C++ a variable must stay consistent when being used in a defined function. This is not the case with Haskell, because you could assign variable ‘x’ to equal 10 in one line then go and change it to say 1 in another line down the program. A topic that was discussed in lecture was the fact that Haskell is a “lazy” programming language. When executing a function you wrote an answer will not be displayed until specifically asked by the user. While typing less in a user created program will theoretically save you time, the adjustment period may prove challenging to experienced programmers. 
\indent Some changes programmers would have to make when transitioning from an interpreter language to Haskell is learning to unlearn the variable naming convention of most modern interpreters. When using Java, a programmer might type the expression ‘int x = 1+1’. In Haskell there is no need for the expression ‘int’. Haskell is smart enough to deduce a variable is supposed to have an integer value when an equation is opposite the equal sign.  

 

\medskip\noindent
To typeset Haskell there are several possibilities. For the example below I took the LaTeX code from \href{https://stackoverflow.com/a/3175141/4600290}{stackoverflow} and the Haskell code from \href{https://hackmd.io/@alexhkurz/HylLKujCP}{my tutorial}.

\begin{lstlisting}
-- run the transition function on a word and a state
run :: (State -> Char -> State) -> State -> [Char] -> State
run delta q [] = q
run delta q (c:cs) = run delta (delta q c) cs
\end{lstlisting}

\medskip\noindent
This works well for short snippets of code. For entire programs, it is better to have external links to, for example, Github or \href{https://replit.com/@alexhkurz/automata01#main.hs}{Replit} (click on the "Run" button and/or the "Code" tab).


\section{Programming Languages Theory}

In this section you will show what you learned about the theory of programming languages.

\section{Project}

Projects in Haskell are much more intensive and require more time to accomplish the task you wish to complete. The biggest piece of advice I suggest would be it is ok to not get it immediately. Haskell is great for new open minded individuals but experienced programmers might struggle to switch gears into this new world. Baseball players struggle to hit golf balls because the swing is so different and it is the exact same concept here. 

\section{Conclusions}\label{conclusions}

So in conclusion we have gone over quite a bit of subjects regarding Haskell and its basics to understand completely before undertaking an assignment of your own. While having previous knowledge and experience is usually a benefit to your cause, Haskell might put that notion to the test. Something as simple as adding positive numbers can cause so many problems to programmers' complicated structured brains.

\begin{thebibliography}{99}
\bibitem[PL]{PL} \href{https://github.com/alexhkurz/programming-languages-2021/blob/main/README.md}{Programming Languages 2021}, Chapman University, 2021.
\bibitem[PL]{https://wiki.haskell.org/Learn_Haskell_in_10_minutes}
\end{thebibliography}

\end{document}





